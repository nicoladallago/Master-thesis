\chapter{Trajectories generator}
\label{trajectoriesGenerator}

An important aspect of this project is the path plannig, because the aim is to map and navigate in a enviroment without a priori information and in complete autonomous. The study of path planning alghoritms and the sensors' fusion to obtain the pose of the vehicle is not part of this thesis, however part of this thesis is to generate a trajectory for the UAV based on the output of a path planning algorithm. In particular, usually exploration algorithms provide only setpoints, not full trajectories \cite{potentialField} \cite{visionBasedMAV}. Is then necessary to provide a tool to generate possible trajectories with constraints in the enviroment and in the dynamic of the vehicle. In this project we implemented the solution proposed in the works \cite{minimumSnap} and \cite{polyTraj}, where the authors, after providing model and control, generate a trajectory composed by piecewise polynomial functions.


\section{Trajectoy definition}
\label{trajDefinition}

In this work, a \textit{setpoint} $\boldsymbol{\sigma}_d$is defined as a position in the space, $\mathbf{x}_d$, and a yaw angle, $\psi_d$, since in the next section we will control four degrees of freedom of the UAV, the position in the space and the yaw angle. We consider the problem of navigate through $m$ setpoints at specific time. A trivial trajectory is one that interpolate the setpoints using straight lines. However, such trajectory is inefficient because it require the quadrotor to come to a stop st each setpoint. This method generates trjectories that smoothly transition throught the setpoints at given times. We write down a trajectory as piecewise polynomial functions of order $n$ over $m$ time intervals

\begin{equation}
	\sigma_d(t)=
	\begin{cases}
		\sum\limits_{i=0}^{n}\sigma_{d,i,1}t^i & t_0 \le t < t_1 \\
		\sum\limits_{i=0}^{n}\sigma_{d,i,2}t^i & t_1 \le t < t_2 \\
		\vdots \\
		\sum\limits_{i=0}^{n}\sigma_{d,i,m}t^i & t_{m-1} \le t < t_m \\
	\end{cases}
	\label{eq:defTraj}
\end{equation}

\noindent where $\sigma_{d,i,j}$ is the coefficient of orden $i$ of the trajectory piece $j$ and $t_k$ is the time that the vehicle has to reach setpoint $k$; $i \in \begin{bmatrix}0 & n\end{bmatrix}$, $j \in \begin{bmatrix}1 & m\end{bmatrix}$, $k \in \begin{bmatrix}1 & m\end{bmatrix}$. The interest is then to minimize a cost function which can be written using these piecewise polynomial.

\begin{alignat}{3}
	\min\qquad & \int_{t_0}^{t_m}\mu_{\mathbf{x}} \Bigg|\Bigg|\frac{d^{k_{\mathbf{x}}}\mathbf{x}_d}{dt^{k_{\mathbf{x}}}} \Bigg|\Bigg|^2 + \mu_{\psi}\Bigg(\frac{d^{k_{\psi}}\psi_d}{dt^{k_{\psi}}}\Bigg)^2dt \\
	\text{subject to}\qquad & \boldsymbol{\sigma}_d(t_i) = \boldsymbol{\sigma}_{d,i}, && i=0,\dots ,m \nonumber \\
	& \frac{d^px_d}{dt^p}\Big|_{t=t_j}=0, && j=0,m; & p=1,\dots ,k_r \nonumber \\
	& \frac{d^py_d}{dt^p}\Big|_{t=t_j}=0, && j=0,m; & p=1,\dots ,k_r \nonumber \\
	& \frac{d^pz_d}{dt^p}\Big|_{t=t_j}=0, && j=0,m; & p=1,\dots ,k_r \nonumber \\
	& \frac{d^p\psi_d}{dt^p}\Big|_{t=tj}=0, && j=0,m; & p=1,\dots ,k_{\psi} \nonumber
	\label{eq:minimizationCost}
\end{alignat}

\noindent where $\mu_{\mathbf{x}}$ and $\mu_{\psi}$ are constants that make the integrand nondimensional. Here $\boldsymbol{\sigma}_d = \begin{bmatrix} x_d & y_d & z_d & \psi_d\end{bmatrix}^T$ and $\boldsymbol{\sigma}_{d,i}=\begin{bmatrix}x_{d,i} & y_{d,i} & z_{d,i} & \psi_{d,i}\end{bmatrix}^T$. We also assume that $t_0=0$ without loss of generality. The first constraint indicates that the result trajectory has to pass through the desire setpoints, while the rest of the constraints impose that all the derivatives at the initial and final point have to be zero (is also possible to set them to a specific value if necessary). By using the same choice of \cite{minimumSnap2}, we decide to minimize the snap for the position ($k_{\mathbf{x}}=4$) and the second derivative of the yaw angle ($k_{\psi}=2$). Now we want to formulate the trajectory generation problem as an optimization of a functional but in a finite dimensional setting. This will keep the computational effort very small to guarantee a real time application. In order to to this, we first write the constants $\boldsymbol{\sigma}_{d,i,j} =\begin{bmatrix}x_{d,i,j} & y_{d,i,j} & z_{d,i,j} & \psi_{d,i,j}\end{bmatrix}^T$ as a $4\cdot n\cdot m\times 1$ vector $\mathbf{c}$ with decision variables $\{x_{d,i,j} , y_{d,i,j} , z_{d,i,j} , \psi_{d,i,j}, i \in \begin{bmatrix}0 & n\end{bmatrix}, j \in \begin{bmatrix}0 & m\end{bmatrix}\}$. The trajectory generation preoblem \eqref{eq:minimizationCost} can be written in the form of a quadratic program (QP):

\begin{alignat}{2}
	\min\qquad & \mathbf{c}^TH\mathbf{c} + f^T\mathbf{c} \\
	\text{subject to}\qquad & A\mathbf{c} \le \mathbf{b} \nonumber \\
	& A_{eq}\mathbf{c} = \mathbf{b}_{eq} \nonumber
	\label{eq:quadraticProgram}
\end{alignat}

\noindent where the objective function will incorporate the minimization of the functional while the constraint can be used to satisfy constraints in the trajectory and it derivatives. A specification of an initial condition, final condition, or intermediate condition on any derivative of the trajectory (e.g. $\frac{d^kx_d}{dt^k}$) can be written as a row of the constraints $A_{eq}\mathbf{c} = \mathbf{b}_{eq}$. If conditions do not need to be
specified exactly then they can be represented with the inequality constraint $A\mathbf{c}\le \mathbf{b}$.

\noindent Moreover, to simplify the problem, can be notice that in the cost function of equation \eqref{eq:minimizationCost}, the four dimensions are indipendent, this means that the problem can be split in four different problems for each dimension. In such a way, the construction of the quadratic problem vectors and matrices will be considerable more simple. Furthermore, is possible to assume that each setpoint starts from $t_0=0$ and ends to $t_j=1$. This because if we define a new time varible such as

\begin{equation}
	\tau = \frac{t-t_{j-1}}{t_j-t_{j-1}}
	\label{eq:simplyTime}
\end{equation}

\noindent the new one dimension position at time $\tau$ become

\begin{equation}
	x(\tau) = x\Big(\frac{t-t_{j-1}}{t_j-t_{j-1}}\Big)	
\end{equation}

\noindent and its derivatives

\begin{align*}
	\frac{d}{dt}x(t) &= \frac{d}{dt}x(\tau) \\
	&= \frac{d}{d\tau}\frac{d\tau}{dt}x(\tau) \\
	&= \frac{1}{t_j-t_{j-1}}\frac{d}{d\tau}x(\tau) \\
	&... \\
	\frac{d^k}{dt^k}x(t) &= \frac{1}{(t_j-t_{j-1})^k}\frac{d^k}{d\tau^k}x(\tau)
\end{align*}

\noindent We can thus solve for each piece of any piece-wise trajectory from $\tau=0$ to $1$, then scale to any $t_0$ to $t_j$.


\section{Optimization of a trajectory between two setpoints}

To make the derivation simpler, is better to start the optimization problem with only two setpoints and in only one dimension. In particular, the cost function become 

\begin{alignat}{2}
	J &= \int_{t_0}^{t_1}\Bigg|\Bigg|\frac{d^kx(t)}{dt}\Bigg|\Bigg|^2dt = \mathbf{c}^TH_{(t_0,t_1)}\mathbf{c} \\
	\text{subject to}\qquad & A\mathbf{c} = \mathbf{b} \nonumber
	\label{eq:oneSetpoint}
\end{alignat}

\noindent We can instead look for the non-dimensionalized trajectory $x(\tau)=c_n\tau^n+c_{n-1}\tau^{n-1}+\dots+c_1\tau+c_0$ where $\tau=\frac{t-t_0}{t_1-t_0}$. Note that this makes $\tau$ range from $0$ to $1$. Let $\mathbf{c}=\begin{bmatrix}c_n & c_{n-1} & \dots & c_1 & c_0\end{bmatrix}^T$.We can wite the cost function $J$ in term of the non-dimensionalized trajectory $x(\tau)$:

\begin{align}
	J &= \int_{t_0}^{t_1}\Bigg|\Bigg|\frac{d^kx(t)}{dt}\Bigg|\Bigg|^2dt \\
	&= \int_0^1\Bigg|\Bigg|\frac{1}{(t_1-t_0)^k}\frac{d^kx(\tau)}{d\tau}\Bigg|\Bigg|^2d(\tau(t_1-t_0)+t_0) \nonumber \\
	&= \frac{t_1-t_0}{(t_1-t_0)^{2k}}\int_0^1\Bigg|\Bigg|\frac{d^kx(\tau)}{d\tau}\Bigg|\Bigg|^2d\tau \nonumber \\
	&= \frac{1}{(t_1-t_0)^{2k-1}}\mathbf{c}^TH_{(0,1)}\mathbf{c} \nonumber \\
	&= \mathbf{c}^T\Bigg(\frac{1}{(t_1-t_0)^{2k-1}}H_{(0,1)}\Bigg)\mathbf{c} \nonumber
\end{align}

\noindent Thus, we want to minimize the cost function

\begin{alignat}{2}
	J &= \mathbf{c}^T\Bigg(\frac{1}{(t_1-t_0)^{2k-1}}H_{(0,1)}\Bigg)\mathbf{c} \\
	\text{subject to}\qquad & A\mathbf{c} = \mathbf{b} \nonumber
\end{alignat}

\noindent To find $H_{(0,1)}$, when $\mathbf{c}'=\begin{bmatrix}c_0 & c_1 & \dots & c_{n-1} & c_n\end{bmatrix}^T$, we can find $H'_{(0,1)}$ with:

\begin{align}
	H'[i,j]_{(t_0,t_11)} &=
	\begin{cases}
		\prod\limits_{z=0}^{k-1}(i-z)(j-z)\frac{t_1^{i+j-2k+1}-t_0^{i+j-2k+1}}{i+j-2k+1} & i\ge k \wedge j\ge k \\
		0 & i<k \vee j<k
	\end{cases} \label{eq:Htranspose} \\
	& i=0,\dots,n,\qquad j=0,\dots,n \nonumber
\end{align}

\noindent However, $\mathbf{c}=\begin{bmatrix}c_n & c_{n-1} & \dots & c_1 & c_0\end{bmatrix}^T$. Reflecting $H'$ from equation \eqref{eq:Htranspose} horizontally and vertcally will give the desire $H$ for the form of $\mathbf{c}$ we desire. The function to minimize is then $\Bigg(\frac{1}{(t_1-t_0)^{2k}}H_{(0,1)}\Bigg)$.

\noindent To find $A$

\begin{align*}
	A\mathbf{c} &= \mathbf{b} \\
	\begin{bmatrix}
		A(t_0) \\
		A(t_1) 
	\end{bmatrix}
	\mathbf{c} &=
	\begin{bmatrix}
		x(t_0) \\
		\vdots \\
		x^{(k-1)}(t_0) \\
		x(t_1) \\
		\vdots \\
		x^{(k-1)}(t_1)
	\end{bmatrix}
\end{align*}

\noindent Note that $A\mathbf{c}$ only contains rows where constraints are specified, if a condition is unconstrained just omit a row. Assuming that every condition is constrained, the general form of $A$ is:

\begin{align}
	A[i,j](t) &= 
	\begin{cases}
		\prod\limits_{z=0}^{i-1}(n-z-j)t^{n-j-i} & n-j \ge i \\
		0 & n-j<i
	\end{cases} \label{eq:matrixA} \\
	& i=0,\dots,r-1, \qquad j=0,\dots,n \nonumber
\end{align}

\noindent where $A[i,j]$ represents the $(n-j)$th coefficient of the $i$th derivative. In the non-dimensionalized case, we have $\tau_0=0$ and $\tau_1=1$:

\begin{equation}
	\begin{bmatrix}
		A(\tau_0) \\
		A(\tau_1) 
	\end{bmatrix}
	\mathbf{c} =
	\begin{bmatrix}
		x(t_0) \\
		\vdots \\
		(t_1-t_0)^{k-1}x^{(k-1)}(t_0) \\
		x(t_1) \\
		\vdots \\
		(t_1-t_0)^{k-1}x^{(k-1)}(t_1)
	\end{bmatrix}
\end{equation}


\section{Optimization of a trajectory between $m+1$ setpoints}

In this section, by recalling what we study in the previous section, is possible to derive the equations to optimize a trajectory for an arbitrary number of setpoints. In particular, we seek the piece-wise trajectory:

\begin{equation}
	x(t)=
	\begin{cases}
		x_1(t), & t_0\le t<t_1 \\
		x_2(t), & t_1\le t<t_2 \\
		\vdots \\
		x_m(t), & t_{m-1}\le t<t_m 	
	\end{cases}
\end{equation}

\noindent and continue to minimize the cost function

\begin{alignat}{2}
	J &= \int_{t_0}^{t_m}\Bigg|\Bigg|\frac{d^kx(t)}{dt}\Bigg|\Bigg|^2dt = \mathbf{c}^TH_{(t_0,t_m)}\mathbf{c} \\
	\text{subject to}\qquad & A\mathbf{c} = \mathbf{b} \nonumber
\end{alignat}

\noindent and again look for the non-dimensionalized trajectory

\begin{align}
	x(\tau) &=
	\begin{cases}
		x_1(\tau) = c_{1,n}\tau^n+\dots+c_{1,0}, & t_0\le t<t_1,\quad \tau=\frac{t-t_0}{t_1-t_0} \\
		x_m(\tau) = c_{m,n}\tau^n+\dots+c_{m,0}, & t_{m-1}\le t<t_m,\quad \tau=\frac{t-t_{m-1}}{t_m-t_{m-1}}
	\end{cases} \\
	& 0\le\tau<1 \nonumber
\end{align}

\noindent Let $\mathbf{c}_z=\begin{bmatrix}c_{z,n} & c_{z,n-1} & \dots & c_{z,1} & c_{z,0}\end{bmatrix}^T$ and $\mathbf{c}=\begin{bmatrix}\mathbf{c}_1^T & \mathbf{c}_2^T & \dots & \mathbf{c}_m^T\end{bmatrix}^T$. Each piece of the trajectory is optimized individually between $\tau_0=0$ and $\tau_1=1$. We want to minimize:

\begin{alignat}{2}
	J &= \int_{t_0}^{t_m}\Bigg|\Bigg|\frac{d^kx(t)}{dt}\Bigg|\Bigg|^2dt \\
	&= \sum\limits_{z=1}^m\int_{t_z-1}^{t_z}\Bigg|\Bigg|\frac{d^kx_z(t)}{dt}\Bigg|\Bigg|^2dt \nonumber \\
	&= \sum\limits_{z=1}^m\int_0^1\frac{t_z-t_{z-1}}{(t_z-t_{z-1})^{2k}}\Bigg|\Bigg|\frac{d^kx_z(\tau)}{d\tau}\Bigg|\Bigg|^2d\tau \nonumber \\
	&= \sum\limits_{z=1}^m \mathbf{c}_z^T\frac{1}{(t_z-t_{z-1})^{2k-1}}H_{(0,1)}\mathbf{c}_z \nonumber \\
	&= \mathbf{c}^TH\mathbf{c} \nonumber \\
	\text{subject to} \qquad & A\mathbf{c}=\mathbf{b} \nonumber
\end{alignat}

\noindent To find $H$, we recall that for each $\mathbf{c}'_z=\begin{bmatrix}c_{z,0} & c_{z,1} & \dots & c_{z,n-1} & c_{z,n} \end{bmatrix}^T$, where $z=1,\dots,m$, $H'_{(0,1)}$ is given by equation \eqref{eq:Htranspose}. Since $\mathbf{c}_z=\begin{bmatrix}c_{z,n} & c_{z,n-1} & \dots & c_{z,1} & c_{z,0}\end{bmatrix}^T$, refletting $H'$ horizzontally and vertically will give the desire $H$ for the form of $\mathbf{c}_k$. Is then possible to create the block diagonal matrix $H$ 

\begin{equation}
	H = 
	\begin{bmatrix}
		\frac{1}{(t_1-t_0)^{2k-1}}H_{(0,1)} & \dots & 0 & 0 \\
		 & \dots & \dots & \vdots \\
		\dots & 0 & \frac{1}{(t_{m-1}-t_{m-2})^{2k-1}}H_{(0,1)} & 0 \\
		0 & \dots & 0 & \frac{1}{(t_m-t_{m-1})^{2k-1}}H_{(0,1)}
	\end{bmatrix}	
\end{equation}

\noindent To find $A$, first, we need to account for endpoint constraints, in the non-dimensionalized case:

\begin{align}
	A_{endpoint}\mathbf{c} &= \mathbf{b}_{endpoint} \\
	\begin{bmatrix}
		A(\tau_0) & 0         & \dots & 0         \\
		A(\tau_1) & 0         & \dots & 0         \\
		0         & A(\tau_0) & \dots & 0         \\
		0         & A(\tau_1) & \dots & 0         \\
		\vdots    & 0         & \dots & \vdots    \\
		0         & \dots     & 0     & A(\tau_0) \\
		0         & \dots     & 0     & A(\tau_1)            
	\end{bmatrix}
	\mathbf{c} &= 
	\begin{bmatrix}
		x_1(t_0) 								\\
		\vdots   								\\
		(t_1-t_0)^{k-1}x_1^{(k-1)}(t_0) 		\\
		x_1(t_1) 								\\
		\vdots   								\\
		(t_1-t_0)^{k-1}x_1^{(k-1)}(t_1) 		\\
		\vdots   								\\
		x_m(t_{m-1}) 							\\
		\vdots                                  \\
		(t_m-t_{m-1})^{k-1}x_m^{(k-1)}(t_{m-1}) \\
		x_m(t_m)                                \\
		(t_m-t_{m-1})^{k-1}x_m^{(k-1)}(t_m)
	\end{bmatrix} \nonumber
\end{align}

\noindent Like before, we just omit rows where a condition is unconstrained. Also, note that except for constraints at $t_0$ and $t_m$, every other constraint must be include twice. The equation for $A[i,j](t)$ is the same of \eqref{eq:matrixA}

\noindent We must also take into account for constraints that ensure that when the trajectory switches from one piece to another at the sepoints, position and all the derivative lower than $k$ remain continuous, for a smooth path. In other words, is require that

\begin{align}
	A_{cont}\mathbf{c} &= \mathbf{b}_{cont} \\
	\begin{bmatrix}
		x_1(t_1)-x_2(t_2) 							  \\
		\vdots            							  \\
		x_1^{(k-1)}(t_1)-x_2^{(k-1)}(t_1) 			  \\
		\vdots                                        \\
		x_{m-1}(t_{m-1})-x_m(t_{m-1})                 \\
		\vdots                                        \\
		x_{m-1}^{(K-1)}(t_{m-1})-x_m^{(K-1)}(t_{m-1})
	\end{bmatrix} 
	&= 0 \nonumber
\end{align}

\noindent Translating to the non-dimeensionalized case, $\tau_0=0$, $\tau_1=1$, and

\begin{align}
	A_{cont}\mathbf{c} &= \mathbf{b}_{cont} \\
	\begin{bmatrix}
		x_1(\tau_1)-x_2(\tau_2) \\
		\vdots \\
		\frac{1}{(t_1-t_0)^{k-1}}x_1^{(k-1)}(\tau_1)-\frac{1}{(t_2-t_1)^{k-1}}x_2^{(k-1)}(\tau_1) \\
		\vdots \\
		x_{m-1}(\tau_1)-x_m(\tau_0) \\
		\vdots \\
		\frac{1}{(t_{m-2}-t_{m-1})^{k-1}}x_{m-1}^{(K-1)}(\tau_1)-\frac{1}{(t_m-t_{m-1})^{k-1}}x_m^{(K-1)}(\tau_0)
	\end{bmatrix}
	&= 0 \nonumber
\end{align}

\noindent and then

\begin{equation}
	\begin{bmatrix}
		A_{cont}(t_1) & 0             & \dots & 0 \\
		0             & A_{cont}(t_2) &	\dots & 0 \\
		\vdots        & 0             & \dots & 0 \\
		0             & \dots         & 0     & A_{cont}(t_{m-1})  
	\end{bmatrix}
	\mathbf{c}=0
\end{equation}

\noindent where

\begin{align}
	A_{cont}[i,j](t_z) &=
	\begin{cases}
		\frac{1}{(t_z-t_{z-1})^i}\prod\limits_{z=0}^{i-1}(n-z-j)\tau_1^{n-j-i}, & n-j\ge i\wedge j\le n \\
		0,                                                                       & n-j<i\wedge j\le n   \\
		-\frac{1}{(t_{z+1}-t_z)^i}\prod\limits_{z=0}^{i-1}(1-z-j)\tau_0^{1-j-i}, & 1-j\ge i\wedge j>n   \\
		0,                                                                       & 1-j<i\wedge j>n      
	\end{cases} \\
	& i=0,\dots,k-1,\quad j=0,\dots,2(n+1) \nonumber 
\end{align}

\noindent The final constraints $A\mathbf{c}=\mathbf{b}$ take then the final form

\begin{align} 
	A\mathbf{c} &= \mathbf{b} \\
	\begin{bmatrix}
		A_{endpoint} \\
		A_{cont}
	\end{bmatrix}
	\mathbf{c} &=
	\begin{bmatrix}
		b_{endpoint} \\
		0
	\end{bmatrix} \nonumber	
\end{align}

\begin{figure}[h]
	\centering
 	\includestandalone[valign=t]{./chapter_4/plots/poly_group_left}
 	\includestandalone[valign=t]{./chapter_4/plots/poly_group_right}
 	\caption{Generated trajectory with its derivatives.}
 	\label{fig:polyTrajectory}		
\end{figure}

\noindent In figure \ref{fig:polyTrajectory} is reported the first dimension of a generated trajectory evolving over the time. In particular in blue are plotted the trajectory and its first four derivatives, till the snap. Instead in black are plotted the setpoints for the trajectory and the initial and final conditions for each derivative, that are equal to zero. Notice that between setpoints two and three, the trajectory is not as expected, in the sense that first tend to go far from the desired setpoint and than reach it. That's because this is only one dimension of a more complex three dimension trajectory.

\section{Adding corridor constraints}

In this section, corridors constraints will be added in the cost function \eqref{eq:minimizationCost}. For corridor constraints, we mean that the desire trajectory must be inside a corridor, this why for a safe obstacle avoidance and navigation algorithm, the vehicle must respect distance between walls and obstacles. To do this, we first define $\mathbf{u}_i$ as the unit vector along the segment from setpoint $\mathbf{r}_i$ and setpoint $\mathbf{r}_{i+1}$.

\begin{equation}
	\mathbf{u}_i=\frac{\mathbf{r}_{i+1}-\mathbf{r}}{||\mathbf{r}_{i+1}-\mathbf{r}||}
	\label{eq:unitVector}
\end{equation}

\noindent The perpendicular distance vector, $\mathbf{d}_i(t)$, from segment $i$ is defined as

\begin{equation}
	\mathbf{d}_i(t) = (\mathbf{r}_d(t)-\mathbf{r}_i)-((\mathbf{r}_d(t)-\mathbf{r}_i)\cdot\mathbf{t}_i)\mathbf{u}_i		
	\label{eq:perpendicularVector}	
\end{equation}

\noindent where $\mathbf{r}_d(t)$ is the desire trajectory at istant $t$. A corridor width on the infinity norm, $\delta_{i}$, is defined for each corridor as follow

\begin{equation}
	\big|\big|\mathbf{d}_i(t)\big|\big|_{\infty}\le\delta_i \quad \text{while} \quad t_i\le t\le t_{i+1}
	\label{eq:corridorWidth}
\end{equation}

\noindent The reason to write the constraint like that, is because it can be incorporate into the QP problem by introducing $n_c$ intermediate points as

\begin{equation}
	\Bigg|x_a\cdot\mathbf{d}_i\Big(t_i+\frac{j}{1+n_c}(t_{i+1}-t_i)\Big)\Bigg|\le\delta_i \quad \text{for} \quad j=1,\dots,n_c
	\label{eq:intermediatePoints}
\end{equation}

\noindent where for $x_a$ we mean that this procedure must be compute for $x_W$, $y_W$ and $z_W$, with $\mathbf{x}_W=\begin{bmatrix}x_W & y_W & z_W\end{bmatrix}^T$. Of course a corridor constraint in the desire yaw doesn't have sense. To do this, we introduced inequality constraints of the form $A_{ineq}\mathbf{c}\le \mathbf{b}_{ineq}$. 

\noindent To find $A_{ineq}$, we first break down the inequality \eqref{eq:intermediatePoints} into

\begin{align}
	(x_a\cdot\mathbf{d}_i(t_i+\frac{j}{1+n_c}(t_{i+1}-t_i)))\le & \delta_i \\
	-(x_a\cdot\mathbf{d}_i(t_i+\frac{j}{1+n_c}(t_{i+1}-t_i)))\le & \delta_i
	\label{eq:breakConstraint}
\end{align}

\noindent This result in a total of $2\cdot2\cdot n_c$ constraints for each corridor constraint. Then by performing some math, the matrix $A_{ineq}$ and the vector $b_{ineq}$ can be deduce \footnote{The formulation of such matrix is very eavy and difficoult to understand, that's why is not report. \textcolor{red}{Or maybe put in a appendix}}.

\begin{figure}[h]
	\centering
 	\includestandalone[]{./chapter_4/plots/corridor}
 	\caption{Setpoints and desire trajectory, with and without corridor constraints between setpoints 2 and 3.}
 	\label{fig:trajectory}		
\end{figure}

\noindent In figure \ref{fig:trajectory} are reported examples of trajectories, with and without corridor constraints. For better understanding are reposrt 2D trajectories, but the same example could be made also for 3D trajectories. The corridor constraint is present only between two setpoints, setpoint 2 and setpoint 3. As is possible to see, the trajectory remain in beetween the constraint, but become less smooth in compare to the one without constraints. Moreover, notice that the entire trajectory is different and not only the segment in between the corridor, this is another important advantage of using this technique to generate trajectores. 
