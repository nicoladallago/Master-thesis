\chapter{Conclusions and future works}
\label{conclusions}

In conclusion, in this thesis we contribute to the development of the control of the Prometheus mapping drone. 

\noindent We first derived a mathematical model of the vehicle with a different structure from any other commercial UAV. Then, we have selected the hardware and software used in the project. 

\noindent We have linearized the system under some assumptions and compute a full estimation of the model using a Kalman filter approach.

\noindent Then we implemented a trajectory generator that smoothly connect the setpoints from navigation or covering algorithms, adding constraints necessary for indoor application.

\noindent In the end we have developed two controllers, one able to track the trajectory previously generated and one to control the attitude of the UAV.

\noindent However, this is just a first approach in the control of the Prometheus UAV. During the development of this thesis project, we encounter many problems and different solutions to them. We applied some solutions but, in general. better performances could be achieve if we used other solutions. In particular, in the model derivation we didn't take into account other phenomena, like aerodynamics turbulence caused by the so called ground effect, the battery drain, and the deformation of the propellers at different speed. We could also have used different strategies for the system identification part, using more complicated algorithms that maybe work better for the full nonlinear system of the quadrotor. The trajectory generator on one hand generates trajectory that are smooth, in the sense that minimize the snap; on the other hand, it doesn't take into account the performance of the UAV. For instance, if we compute a trajectory too aggressive and infeasible, we could end up with a big error in the control. Further investigations are needed in this case. However, we could compute a trajectory that take into account also the dynamic, for instance by adding other constraints or by using a different cost function. 

\noindent For the control part, we add some simplifications in the dynamic of the moving sensors, adding errors in the steady state. We could also have computed a mathematical model for the cart only, from the motor signal to the dynamic in the ring, then coupling together with the model and obtained a more sophisticated controller. Moreover, in the MPC, we didn't investigate what all the linearization of the rotation matrices actually comport, and we didn't exploit other solutions, like \textit{non-linear MPC}, \textit{adaptive MPC} or switching MPC also for the rotations and position. Another interesting solution for the control part, could be to run the system identification online, and take into account also of the varying of the parameters, derive from effects like the battery drain during the fly.