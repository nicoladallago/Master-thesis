\chapter{Control}
\label{control}

The last important step of this work, is to derive a control law, able to track the desire trajectory and to compensate for the movement of the sensor. We chose two different control strategies, that are videly used in the control of quadrotor tyoe UAV \cite{LeeController}, \cite{MPCGeorge}.

\section{Position tracking controller in \texorpdfstring{$SO(3)$}{TEXT}}

In this section, will be introduce a tracking control for the vehicle, based on the work \cite{LeeController}. We will use this control because it has been show to work well in many different applications, from precision flights, to fast and aggressiv flights. In particular, starting from the model of the UAV, we will derive a position tracking control, based on $SO(3)$ group.

\noindent The \textit{general linear group} of order $3$, $GL(3)$, is a algebraic group composed by all the non singular matrices $A \in \rm I\!R^{3\times 3}$ with matrix product.

\noindent The \textit{orthogonal group} of order $3$, $O(3)$, is deefine as

\begin{equation*}
	O(3) = \{A \in GL(3) : A^TA=I \} \subseteq GL(3)
\end{equation*}

\noindent Is easy to prove that if $A$ belongs to $O(3)$, then $\det[A]=\pm 1$.

\noindent The \textit{special orthogonal grouop} of order $3$, $SO(3)$, is define as

\begin{equation*}
	SO(3) = \{A \in O(3) : \det[A] = 1\} 
\end{equation*}

\noindent The rotation matrices belong to this gruop, and it is because the controller will use errors based in the rotation matrices, we need to keep all the results in the group $SO(3)$.

\noindent To derive the control law, we first need to define the tracking errors. In particular the position and the velocity tracking errors are given by, respectively

\begin{align}
	\mathbf{e}_x &= \mathbf{x} - \mathbf{x}_d \\ 
	\mathbf{e}_v &= \dot{\mathbf{x}} - \dot{\mathbf{x}}_d
\end{align}

\noindent where the subscript $d$ stands for desire. The attitude error is insted define as

\begin{equation}
	\mathbf{e}_R = \frac{1}{2}\bigl(R_c^TR-R^TR_C \bigl)^{\vee}
\end{equation}

\noindent where $R$ is the rotation matrix that encode the actual attitude of the UAV, while $R_c(t) \in SO(3)$ is the computed attitude matrix, that must belongs to the special ortoghonal group. In fact we can define it as

\begin{align}
	\mathbf{b}_{1,c} &= \begin{bmatrix} \cos(\psi_d) & \sin(\psi_d) & 0 \end{bmatrix}^T \\
	\mathbf{b}_{3,c} &= -\frac{k_x\mathbf{e}_x+k_v\mathbf{e}_v-g\mathbf{e}_3-\ddot{\mathbf{x}}_d}{||k_x\mathbf{e}_x+k_v\mathbf{e}_v-g\mathbf{e}_3-\ddot{\mathbf{x}}_d||} \\
	\mathbf{b}_{2,c} &= \frac{\mathbf{b}_{3,c}\times\mathbf{b}_{1,c}}{||\mathbf{b}_{3,c}\times\mathbf{b}_{1,c}||} \\
	R_c &= \left[\begin{array}{c|c|c}\mathbf{b}_{2,c}\times\mathbf{b}_{3,c} & \mathbf{b}_{2,c} & \mathbf{b}_{3,c} \end{array}\right]
\end{align}

\noindent Where $\psi_d$ is the desire yaw, $\mathbf{e}_3$ is the third dimension canonical vector, $k_x$ and $k_v$ are positive control constants. The \textit{vee} map $\cdot^{\vee}$ is the inverse of the \textit{hat} map $\hat{\cdot}: \rm I\!R^3 \rightarrow SO(3)$ define as

\begin{align}
	\mathbf{v} &= \begin{bmatrix} v_1 & v_2 & v_3 \end{bmatrix}^T \nonumber \\ 
	\hat{\mathbf{v}} &= 
	\begin{bmatrix}
		0    & -v_3 & v_2 \\
		v_3  & 0    & -v_1 \\
		-v_2 & v_1  & 0
	\end{bmatrix}
\end{align}

\noindent The angular velocity error $e_{\omega}$ depends only from the desire yaw, since our trajectory generator compute the desire yaw till the second derivative. However, it can be compute simple by 

\begin{equation}
	\mathbf{e}_{\omega} = \boldsymbol{\omega} - R^TR_C\hat{\boldsymbol{\omega}}_c
\end{equation}

\noindent where $\hat{\boldsymbol{\omega}}_c = R_c^T\dot{R}_c$.

\noindent The final control law is then divide in two contributions, one for the total force and one for the torque

\begin{align}
	f &= -(k_x\mathbf{e}_x+k_v\mathbf{e}_v-g\mathbf{e}_3-\ddot{\mathbf{x}}_d)^TR\mathbf{e}_3 \label{eq:LeeController1} \\
	\boldsymbol{\tau} &= -k_R\mathbf{e}_R - k_{\omega}\boldsymbol{e}_{\omega} + \boldsymbol{\omega}\times I_{cm}\boldsymbol{\omega} \label{eq:LeeController2} 
\end{align}

\noindent where $k_R$ and $k_{\omega}$ are again positive control gains. As is possible to see, this controller is very simple to implement and can be prove that the control is exponentialy stable, if the initial conditions are sufficently small\footnote{see the paper \cite{LeeController} for more details and the prooof}. Of course the control law is not complete, because what we can control are the inputs to the motors and not the force and torques. So for computing the linear acceleration in the body frame we just need to do 

\begin{equation}
	\ddot{\mathbf{x}}_B = \frac{1}{m}\cdot\begin{bmatrix}0 & 0 & f\end{bmatrix}^T
\end{equation}

\noindent For the angular acceleration in the body frame, we just need to compute

\begin{equation}
	\dot{\boldsymbol{\omega}}_B = I_{cm}^{-1}\boldsymbol{\tau}
\end{equation}

\noindent Then, by using the results from section \ref{quadModel}, we can do the computation of the matrix that encode the law from motors' inputs to linear and angular acceleration and obtain

\begin{equation}
	\begin{bmatrix}
		\vdots \\
		u_i^2 \\
		\vdots
	\end{bmatrix}
	=
	\underbrace{
	\begin{bmatrix}
		\dots   & A_F   & \dots  \\
		\dots & \Big(\mathbf{l}_i + \boldsymbol{\Delta l}\Big)\times\begin{bmatrix}0 \\ 0 \\ A_F \end{bmatrix}+\begin{bmatrix}0 \\ 0 \\ B_F\end{bmatrix} & \dots
	\end{bmatrix}^{-1}
	}_{T^{-1}(\boldsymbol{\beta})}	
	\cdot
	\begin{bmatrix}
		f \\
		\boldsymbol{\tau}
	\end{bmatrix}
	\label{eq:mixingMatrix}
\end{equation}

\noindent where the matrix $T^{-1}(\boldsymbol{\beta})$ is the inverse of the estimated parameters, except that is encoded only one row for the force, since thre propellers are parallel to the frame of the quadrotor. Mowever the problem is in the computation of the inverse of the inertia matrix, since the estimated parameters are coupled with the component of the inertia and is not possible to etimate directly $A_F$ and $B_F$. A naive approach is simply to use the inertia computed with the CAD model. Of course, this approach will add errors, since the CAD doesn't provide a perfect data. Hovewer, this inverse will be multiply with $k_R$ and $k_{\omega}$ and then just a simple retuning of the parameters will be necessary. Instead, the term $I_{cm}^{-1}\bigl(\boldsymbol{\omega}\times I_{cm}\boldsymbol{\omega}\bigl)$ as said in section \ref{linearization} is very small and will not introduce significant errors.


\subsection{Adding the rotating platform}

\begin{figure}[h]
\centering
	\begin{tikzpicture}[thick,scale=0.85, every node/.style={scale=0.85}]
		\coordinate (origin)  at (0,    0);
		\coordinate (traj)    at (0.5,    1);
		\coordinate (control) at (5,    1);
		\coordinate (mixing)  at (9,    1.5);
		\coordinate (dynamic) at (13,   1);
		\coordinate (output1) at (16.5, 2.5);
		\coordinate (outfit1) at (16,   2.5);
		\coordinate (outfit2) at (4.5,    0.5);
		\coordinate (outfit3) at (4.55,    0.45);

		\node[draw, minimum width=2.5cm, minimum height=3cm, anchor=south west, align=center] (TRA) at (traj) {Trajectory\\generator};
		\node[draw, minimum width=2.5cm, minimum height=3cm, anchor=south west, align=center] (CON) at (control) {Position\\controller\\\eqref{eq:LeeController1}\eqref{eq:LeeController2}};
		\node[draw, minimum width=2.5cm, minimum height=2cm, anchor=south west, align=center] (MIX) at (mixing) {$T^{-1}(\boldsymbol{\beta})$\\\eqref{eq:mixingMatrix}};
		\node[draw, minimum width=2.5cm, minimum height=3cm, anchor=south west, align=center] (DYN) at (dynamic) {Quadrotor\\dynamic\\\eqref{eq:finalModelCart}};

		\draw[->] ($(TRA.0)+(0, 1)$) -- node[above]{$\mathbf{x}_d$, $\dot{\mathbf{x}}_d$, $\ddot{\mathbf{x}}_d$} ($(CON.180)+(0, 1)$);
		\draw[->] ($(TRA.0)-(0, 1)$) - ++(0.75, 0) -- ++(0.75, 0) |- node[above]{\hspace{15pt}$\psi_d$, $\dot{\psi}_d$, $\ddot{\psi}_d$} ($(CON.180)$);
		\draw[->] ($(CON.0)+(0, 1)$) - ++(0.75, 0)node[above]{$f$} -- ++(0.75, 0) |- ($(MIX.180)+(0, 0.5)$);
		\draw[->] ($(CON.0)-(0, 1)$) - ++(0.75, 0) -- ++(0.75, 0) |- node[above]{$\boldsymbol{\tau}$} ($(MIX.180)-(0, 0.5)$);
		\draw[->] ($(MIX.0)$) -- node[above]{$\mathbf{u}$} ($(DYN.180)$);
		\draw[->] ($(DYN.0)+(0, 1)$) -- node[above]{\hspace{5pt}$\mathbf{x}$, $\dot{\mathbf{x}}$, $\ddot{\mathbf{x}}$} ($(DYN.0)+(1, 1)$);
		\draw[->] ($(DYN.0)-(0, 1)$) -- node[above]{\hspace{10pt}$R$, $\boldsymbol{\omega}$, $\dot{\boldsymbol{\omega}}$} ($(DYN.0)+(1, -1)$);
		\draw[-] ($(DYN.0)+(0.5, -1)$) |- (outfit2);
		\draw[-] ($(DYN.0)+(0.55, 1)$) |- (outfit3);
		\draw[->] (outfit2) |- ($(CON.180)-(0, 1)$);
		\draw[->] (outfit3) |- ($(CON.180)-(0, 1.05)$);
		\draw[->] ($(MIX.90)+(0, 0.8)$) -- node[right]{$\gamma$} ($(MIX.90)$);
		%\draw[-] ($(MIX.90)+(0, 0.8)$) -- ($(CON.90)+(0, 0.3)$);
		%\draw[->] ($(CON.90)+(0, 0.3)$) -- ($(CON.90)$);
 
	\end{tikzpicture}
	\caption{Block diagram of the control scheme (the subscripts that indicate the appartenence of the frame are omitted).}
	\label{fig:controlBlock}
\end{figure}

\noindent The previous controller was derive without the roating platform. To introduce the compensation for the movement of the cart, first of all we introduce the compensation for the moving CoG. To do this we simply modify the terms $\mathbf{l}_i$ in the matrix $T(\boldsymbol{\beta})$ and then compute the inverse at every iteration. Of course, to do this, we need to know precisely the position $\gamma$ of the cart, this can be done since the motor that drive the cart is provided of encoder \cite{Carlos}. The changing in the inertia matrix are instead compute with the CAD, a more preciselly solution could be to compute the system identification with the moving cart or, if if doesn't work, compute the system identification multiple times with different positions of the sensors and then interpret the system as a piecewise system. 

\noindent The effects coming from the centrifugal force are not compensate in this controller, becasue the maximum speed of the cart is sufficiently small to assume that all this effects are neglectable. However, in case we want to take into account those, we just need to add the compensation in the force vector, using the equation \eqref{eq:cartAllForces}.

\subsection{Simulation results}

\begin{figure}[h]
	\centering
 	\includestandalone[]{./chapter_5/plots/simulation1}
 	\caption{Simulation results to a step input in the desire $z$, then hovering. Without and with the cart compensation. The mass of the moving cart is one third of the real mass. The cart is moving with a constant speed of $6$ rpm, the maximum possible for the engine adopted \cite{Carlos}.}
 	\label{fig:simulation1}		
\end{figure}

\noindent In figure \ref{fig:simulation1} is reported a simple example of trajectory tracking. In particular, the desire trajectory is a step for the desire height $z$ and hovering for the remaining variables. Moreover is illustrate why the cart compensation is important. 

\noindent In red are depicted the result trajectories with the compensation, while, in blue, without the compensation. Note how the results is much more stable with the compensation. Note also that in the simulation the mass of the moving cart is one third of the real mass, this because if we simulate without compensation, the controller is not able to keep the vehicle balance, resulting in a complete loss of control. 

\noindent In figure \ref{fig:simulation2} is instead depicted a simulation with the full mass of the cart. The control algorithm is the one with the cart compensation and the desire trajectory is again a step imput for the desire height and constant for the other inputs. As is possible to see the control works quite well but some oscillations are again noticeable. Those oscillations come from the engines. It means that we know the position of the cart at the actual time and than apply the control compensation. However the engines apply a delay to the system and this compensation is delayed by the time constant of the motors. 

\begin{figure}[h]
	\centering
 	\includestandalone[]{./chapter_5/plots/simulation2}
 	\caption{Simulation results to a step input in the desire $z$, then hovering. The cart is moving with a constant speed of $6$ rpm, the maximum possible for the engine adopted \cite{Carlos}.}
 	\label{fig:simulation2}		
\end{figure}

\noindent This delay is very clear especially in the $y$ position, since the moving part is in the $y$ and $z$ axis. A solution for this problem could be to estimate the position of the cart in advance with an accurate model of this. Others problems are some bias errors in the steady state, however this problems come from the simulated bias of the sensors. These errors are in conclusion in the order of the millimiter, and no solutions are provide in this thesis, but is a direction for a future work in this project.

\begin{figure}[h]
	\centering
	\hspace{11pt}\includestandalone[]{./chapter_5/plots/simulation3_z}
 	\hspace{-5pt}\includestandalone[]{./chapter_5/plots/simulation3_xy}
 	\caption{Simulation of trajectory tracking.}
 	\label{fig:trajectoryTrackingLee}		
\end{figure}

\noindent In figure \ref{fig:trajectoryTrackingLee} is instead reported an example of trajectory tracking with a smooth generated trajectory.


\section{Model predictive control}

\textit{Model predictive control} (MPC) is an advanced method of process control that has been in use in the process industries in chemical plants and oil refineries since the 1980s. In recent years it has also been used in power system balancing models. Model predictive controllers rely on dynamic models of the process, most often linear empirical models obtained by system identification. The main advantage of MPC is the fact that it allows the current timeslot to be optimized, while keeping future timeslots in account. This is achieved by optimizing a finite time-horizon, but only implementing the current timeslot. MPC has the ability to anticipate future events and can take control actions accordingly. PID and LQR controllers do not have this predictive ability. MPC is nearly universally implemented as a digital control, although there is research into achieving faster response times with specially designed analog circuitry\footnote{\url{https://en.wikipedia.org/wiki/Model_predictive_control}}.

\noindent MPC is based on iterative, finite-horizon optimization of a model. At time $t$ the current plant state is sampled and a cost minimizing control strategy is computed (via a numerical minimization algorithm) for a relatively short time horizon in the future. Model Predictive Control is a multivariable control algorithm that uses:

\begin{itemize}
	\item an internal dynamic model of the process
	\item a history of past control moves and
	\item an optimization cost function $J$ over the receding prediction horizon,
\end{itemize}

\noindent to calculate the optimum control moves. In particular, we use a state space model like

\begin{equation}
	\mathbf{x}(t+1) = A\mathbf{x}(t) + B\mathbf{u}(t)
\end{equation}

\noindent where $\mathbf{x}(t) \in \rm I\!R^n$, $\mathbf{u}(t) \in \rm I\!R^m$ are the state and input vectors, respectively, subject to the constraints 

\begin{equation}
	\mathbf{x}(t) \in X, \mathbf{u}(t) \in U, \forall t \ge 0
\end{equation}  

\noindent where the sets $X \subseteq \rm I\!R^n$ and $U \subseteq \rm I\!R^m$ are polyhedra \cite{MPC}.MPC approaches such a constrained regulation problem in the following way. Assume that a full measurement or estimate of the state $\mathbf{x}(t)$ is available at the current time $t$. Then the finite time optimal control problem

\begin{align}
	&\min_{U_{t\to t+N|t}} &&J_t\big(\mathbf{x}(t), U_{t\to t+N|t}\big)=\sum_{i=1}^N||\mathbf{r}_i-\mathbf{x}_i||^2_{W_x} + \sum_{i=1}^N||\boldsymbol{\Delta}\mathbf{u}||^2_{W_u} \label{eq:MPC} \\
	&\text{subject to} &&\mathbf{x}_{t+k+1|t}=A\mathbf{x}_{t+k|t}+B\mathbf{u}_{t+k|t}, \quad k=1,\dots,N \nonumber \\
	& &&\mathbf{x}_{t+k|t} \in X, \quad \mathbf{u}_{t+k|t} \in U, \quad k=1,\dots,N \nonumber \\
	& &&\mathbf{x}_{t|t}=\mathbf{x}(t) \nonumber
\end{align}

\noindent where $||\cdot||^2_{W}$ is the square weighted norm, with weight matrix $W$. Now MPC just solve the optimization problem \eqref{eq:MPC} and obtain the corresponding input signal. The prediction of the state will be obtain with a standard Kalman filter\footnote{For further more details, see \cite{MPC}}. In conclusion, the main advantages of MPC for our applications are:

\begin{itemize}
	\item take into account of all the system, include the motors,
	\item constraints in the input, in our case $u_{i} \in \begin{bmatrix}0 & 1 \end{bmatrix}$
\end{itemize}

\noindent while the disvantages are:

\begin{itemize}
	\item require the linearization of the system, loosing precision.
	\item more computation effort for the onboard CPU.
\end{itemize}

\subsection{Linear model}

First of all we need to compute a linear model for our vehicle. In particular, if we use the results from section \ref{linearization} we already have a linear model like

\begin{align}
	\dot{\boldsymbol{\omega}}_B &= A(\boldsymbol{\beta})\mathbf{u} \label{eq:MPCdyn1} \\
	\ddot{\mathbf{x}}_B &= L(\beta_1)\mathbf{u} \label{eq:MPCdyn2} \\
	\dot{u} &= \underbrace{-\frac{1}{\tau}I_4}_{=\Delta_A}\mathbf{u} + \underbrace{\frac{1}{\tau}I_4}_{=\Delta_B}\mathbf{u}_{in}^2 \label{eq:MPCdyn3}
\end{align}

\noindent Now we need a linear model also for the position and orientation. To do this, we will use the Euler angles instead of the quaternions, because all the math will be more intuitive and clean, however, same results are possible using quaternions. If we denote the orientation in the world frame as $\boldsymbol{\theta}_W = \begin{bmatrix} \phi & \theta & \psi \end{bmatrix}^T$ we have that

\begin{equation}
	\dot{\boldsymbol{\theta}}_W = T\boldsymbol{\omega}_B
	\label{eq:thetaDynamic}
\end{equation}

\noindent where $T$ is the rotation matrix

\begin{equation}
	T = 
	\begin{bmatrix}
		1 & \sin(\phi)\tan(\theta)          & \cos(\phi)\tan(\theta) \\
		0 & \cos(\phi)                      & -\sin(\phi)            \\
		0 & \frac{\sin(\phi)}{\cos(\theta)} & \frac{\cos(\phi)}{\cos(\theta)} 
	\end{bmatrix}
\end{equation}

\noindent This is a higly non-linear system. To make this linear we chose the approach of the small angles simplification. It means that if we assume that the angles are small enough, we can approximate the trigonometric functions with the Taylor expansion:

\begin{align*}
	\sin(x) &\approx x \\
	\cos(x) &\approx 1-\frac{x^2}{2} \\
	\tan(x) &\approx x \\
\end{align*}

\noindent Moreover, we assume that the multiplication between two angles is approximate equal to zero. From this, coupling equation \eqref{eq:thetaDynamic} and the simplifications we have

\begin{equation}
	\dot{\boldsymbol{\theta}}_W = T\boldsymbol{\omega}_B \approx 
	\begin{bmatrix}
		1 & 0      & \theta \\
		0 & 1      & -\phi  \\
		0 & \theta & 1 \\ 
	\end{bmatrix}
	\begin{bmatrix}
		\omega_{x,B} \\
		\omega_{y,B} \\
		\omega_{z,B}
	\end{bmatrix} = 
	\begin{bmatrix}
		\omega_{x,B}+\theta\omega_{z,B} \\
		\omega_{y,B}-\phi\omega_{z,B}   \\
		\omega_{z,B}+\phi\omega_{z,B}
	\end{bmatrix} \approx
	\begin{bmatrix}
		\omega_{x,B} \\
		\omega_{y,B} \\
		\omega_{z,B}
	\end{bmatrix}
	= \boldsymbol{\omega}_B
	\label{eq:MPCdyn4}
\end{equation}

\noindent Similar reasoning can be apply to the position in the world frame, that is

\begin{equation}
	\dot{\mathbf{x}}_W = R\dot{\mathbf{x}}_B
\end{equation}

\noindent where $R$ is the rotation matrix, (for simplify the notation, $s(x)=\sin(x)$ and $c(x)=\cos(x)$)

\begin{equation*}
	R = 
	\begin{bmatrix}
		c(\psi)c(\theta) & -s(\psi)c(\phi)+c(\psi)s(\theta)s(\psi) & s(\psi)s(\phi)+c(\psi)s(\theta)c(\psi) \\
		s(\psi)c(\theta) & c(\psi)c(\phi)+s(\psi)s(\theta)s(\psi) & -c(\psi)s(\phi)+s(\psi)s(\theta)c(\psi) \\
		s(\theta) & c(\phi)s(\theta) & c(\theta)c(\phi)
	\end{bmatrix}
\end{equation*} 

\noindent and appling again the small angles approximations, we have

\begin{align}
	\dot{\mathbf{x}}_W = R\dot{\mathbf{x}}_B &\approx
	\begin{bmatrix}
		1 & -\psi & \theta \\
		\psi & 1 & -\phi \\
		-\theta & \phi & 1
	\end{bmatrix}
	\begin{bmatrix}
		\dot{x}_{x,B} \\
		\dot{x}_{y,B} \\
		\dot{x}_{z,B} 
	\end{bmatrix} \nonumber \\
	&= 
	\begin{bmatrix}
		\dot{x}_{x,B}-\psi\dot{x}_{y,B}+\theta\dot{x}_{z,B} \\
		\psi\dot{x}_{x,B}+\dot{x}_{y,B}-\phi\dot{x}_{z,B} \\
		-\theta\dot{x}_{x,B}+\phi\dot{x}_{y,B}+\dot{z}_{x,B}
	\end{bmatrix}
	\approx
	\begin{bmatrix}
		\dot{x}_{x,B} \\
		\dot{x}_{y,B} \\
		\dot{x}_{z,B}
	\end{bmatrix}
	= 
	\dot{\mathbf{x}}_B
	\label{eq:MPCdyn5}
\end{align}

\noindent The last simplification, actualy is true if we consider the position error, and not the position, if we assume that the error is suffucuently small. This makes sense, since in our application we have that the trajectory is computed smooth and without any step or artifacts that make the error big. Now, we need to add the gravity term and in particular we have

\begin{align}
	\ddot{\mathbf{x}}_B = R^T
	\begin{bmatrix}
		0 \\
		0 \\
		-g
	\end{bmatrix}
	+L(\beta_1)\mathbf{u} &\approx
	\begin{bmatrix}
		1 & \psi  & -\theta \\
		-\psi & 1 & \phi \\
		\theta & -\phi & 1  
	\end{bmatrix}
	\begin{bmatrix}
		0 \\
		0 \\
		-g
	\end{bmatrix}	
	+L(\beta_1)\mathbf{u} \nonumber \\
	&=
	\begin{bmatrix}
		-\theta g \\
		\phi g \\
		-g 
	\end{bmatrix}
	+L(\beta_1)\mathbf{u} \nonumber \\
	&=\underbrace{ 
	\begin{bmatrix}
		0 & -g & 0 \\
		g & 0  & 0 \\
		0 & 0  & 0
	\end{bmatrix}}_{=A_g} + L(\beta_1)\mathbf{u}+
	\begin{bmatrix}
		0 \\
		0 \\
		-g
	\end{bmatrix}
	\label{eq:MPCdyn6}	
\end{align}

\noindent Coupling togheter equations \eqref{eq:MPCdyn1}, \eqref{eq:MPCdyn2}, \eqref{eq:MPCdyn3}, \eqref{eq:MPCdyn4}, \eqref{eq:MPCdyn5}, and \eqref{eq:MPCdyn6}, we obtain a linear model for the control of the UAV:

\begin{equation}
	\underbrace{
		\begin{bmatrix}
			\dot{\boldsymbol{\omega}}_B \\
			\dot{\boldsymbol{\theta}}_W \\
			\dot{\mathbf{u}}            \\
			\dot{\mathbf{x}}_W          \\
			\ddot{\mathbf{x}}_B
		\end{bmatrix}
	}_{=\dot{\mathbf{x}}} =
	\underbrace{
	\begin{bmatrix}
		\mathbf{0}_{3\times 3} & \mathbf{0}_{3\times 3} & A(\boldsymbol{\beta})  & \mathbf{0}_{3\times 3} & \mathbf{0}_{3\times 3} \\
		I_3                    & \mathbf{0}_{3\times 4} & \mathbf{0}_{3\times 4} & \mathbf{0}_{3\times 3} & \mathbf{0}_{3\times 3} \\
		\mathbf{0}_{4\times 3} & \mathbf{0}_{4\times 3} & \Delta_A               & \mathbf{0}_{4\times 3} & \mathbf{0}_{4\times 3} \\
		\mathbf{0}_{3\times 3} & \mathbf{0}_{3\times 3} & \mathbf{0}_{3\times 4} & \mathbf{0}_{3\times 3} & I_3                    \\
		\mathbf{0}_{3\times 3} & A_g                    & L(\beta_1)             & \mathbf{0}_{3\times 3} & \mathbf{0}_{3\times 3}
	\end{bmatrix}
	}_{=A}
	\underbrace{
		\begin{bmatrix}
			\boldsymbol{\omega}_B \\
			\boldsymbol{\theta}_W \\
			\mathbf{u}            \\ 
			\mathbf{x}_W          \\
			\dot{\mathbf{x}}_B
		\end{bmatrix}
	}_{=\mathbf{x}} + 
	\underbrace{
	\begin{bmatrix}
		\mathbf{0}_{3\times 4} \\
		\mathbf{0}_{3\times 4} \\
		\Delta_B               \\
		\mathbf{0}_{3\times 4} \\
		\mathbf{0}_{3\times 4}
	\end{bmatrix}
	}_{=B} u_{in}^2
	+\underbrace{
	\begin{bmatrix}
		\mathbf{0}_{15\times 1} \\
		-g
	\end{bmatrix}}_{E}
\end{equation} 

\noindent and if we want to control only the position and orinetation, we can select as \textit{measured output} $\boldsymbol{\theta}_W$

\begin{equation}
	\underbrace{\boldsymbol{\theta}_W}_{=\mathbf{y}} = 
	\underbrace{\begin{bmatrix}
		\mathbf{0}_{3\times 3}  & I_3 & \mathbf{0}_{3\times 10} \\
		\mathbf{0}_{3\times 10} & I_3 & \mathbf{0}_{3\times 6}
	\end{bmatrix}}_{=C} \mathbf{x} + \underbrace{\mathbf{0}_{6\times 4}}_{=D}u_{in}^2
\end{equation}

\noindent Finally, we need to discretize the system. We used the built-in \textit{MATLAB} software to do this, because it is easier and faster than do this by hands.

\subsection{Adding the rotating platform} 

The last step of the controller is to add the compensation for the movement of the cart. Again, we take into account only the displacement of the CoG and the changes in the inertia matrix, neglecting the effects from the centrifugal force. We used a \textit{switching MPC} strategy. We design different linear models that correspond to a different position of the cart. Then, during the cart movement, the controller decide wich linear model is closer to the actual position of the cart. Then apply the MPC regurds that specific linear model.

\subsection{Simulation results}