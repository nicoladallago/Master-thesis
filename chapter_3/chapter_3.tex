\chapter{System identification}
\label{systemIdentification}

In this chapter, is about to be addressed an important part of this project. Since the model of the previous chapter is depending from many parameters, is necessary to identificate them, to be able to design an appropriate controller. A Kalman Filter approach will be used, based from the work \cite{modelIdentification}.

\section{System simplification and linear approximation}
\label{linearization}

Starting from the model deducted in section \ref{quadModel}

\begin{equation}
	\begin{split}
		\begin{bmatrix}
			\ddot{\mathbf{x}}_B \\
			\dot{\boldsymbol{\omega}}_B
		\end{bmatrix}
		&=
		\begin{bmatrix}
			\dots & \frac{a_{f,i}\Omega_{max,i}^2\mathbf{n}_i}{m} & \dots \\
			\dots & I_{cm}^{-1}\Big[ (\mathbf{l}_i+\boldsymbol{\Delta l})\times a_{f,i}\Omega^2_{max,i}\mathbf{n}_i-\sgn(\Omega_i)b_{f,i}\Omega_{max,i}^2\mathbf{n}_i\Big] & \dots
		\end{bmatrix}
		\begin{bmatrix}
			\vdots \\
			u_i^2 \\
			\vdots
		\end{bmatrix}
		+ \\
		&+
		\begin{bmatrix}
			\mathbf{0} \\
			I_{cm}^{-1}\bigl(\boldsymbol{\omega}_B \times I_{cm} \boldsymbol{\omega}_B \bigl)
		\end{bmatrix} \\
	\end{split}
	\label{eq:finalModel1a}
\end{equation}

\begin{equation}
	u_i = \frac{1}{\tau_is+1}u_{in,1}
	\label{eq:finalModel2a}
\end{equation}

\noindent we need to do some simplification. In particular, by assuming that all engines have the same parameters, is possible to rewrite these parameters as follows

\begin{align}
	a_{f,i}\Omega_{max,i}^2 &\approx a_f \\
	b_{f,i}\Omega_{max,i}^2 &\approx b_f \\
	\tau_i &\approx \tau
	\label{eq:simplification}
\end{align}

\noindent Moreover the term $I_{cm}^{-1}\bigl(\boldsymbol{\omega}\times I_{cm}\boldsymbol{\omega}_B\bigl)$ can be neglected \cite{modelIdentification}. This can be easily seen simply by simulating the mathematical model with and without the term, the differences are very small, as depicted in figure \ref{fig:comparisonOmega}. 

\begin{figure}[h]
	\centering
	\includestandalone[]{./chapter_3/plots/comparison}
	\caption{Simulation of the dynamic with and without the term $I_{cm}^{-1}\bigl(\boldsymbol{\omega}\times I_{cm}\boldsymbol{\omega}_B\bigl)$.}
	\label{fig:comparisonOmega}
\end{figure}

\noindent Another simplification, is that the inertia matrix $I_{cm}$ is a diagonal matrix, $I_{cm} = \diag(\allowbreak I_{xx}, I_{yy}, I_{zz})$. This is generally true in a standard quadrotor but is not so immediate for the vehicle of this project. However, if we align the $x$ axis with the orientation of the circular structure, we obtain a inertia matrix almost diagonal. What makes the matrix "less diagonal" is the position of the cart. However, the mass of the sensor is not sufficiently big to modify enough the matrix and this assumption is valid also here. Of course, in different applications, where the mass of the quadrotor and the mass of the sensor are more similar, a different approach is required. 

\noindent Another non linearity is in the inputs, since the model require the square of these. A solution of this problem proposed in \cite{modelIdentification} is to rewrite equation \eqref{eq:finalModel2a} with the square of the control inputs. This effectively moves the squared control signal from the force and torque equations to the input. This representation keeps the static relationship but will affect the dynamics of the first order system, but is assumed that a first order system still captures the majority of the dynamics. In conclusion, the approximate linear model is

\begin{gather}
	\begin{bmatrix}
		\ddot{\mathbf{x}}_B \\
		\dot{\boldsymbol{\omega}}_B
	\end{bmatrix}
	=
	\begin{bmatrix}
		\dots & \frac{a_f \mathbf{e}_3}{m} & \dots \\
		\dots & I_{cm}^{-1}\Big[(\mathbf{l}_i+\boldsymbol{\Delta l})\times a_f\mathbf{e}_3-\sgn(\Omega_i)b_f\mathbf{e}_3\Big] & \dots 
	\end{bmatrix}
	\begin{bmatrix}
		\vdots \\
		u_i \\
		\vdots
	\end{bmatrix} 
	\label{eq:linearModel1}\\
	u_i = \frac{1}{\tau s+1}u_{in,i}^2
	\label{eq:linearModel2}
\end{gather}

\noindent where instead of $\mathbf{n}_i$ there is $\mathbf{e}_3$ because in the structure of this particular vehicle, the propellers are mounted parallel to the ground and then with a force vector aligned to $\mathbf{e}_3=\begin{bmatrix}0 & 0 & 1\end{bmatrix}^T$.


\section{Quadrotor parameters}
\label{quadParameters}

From the simplified model of equations \eqref{eq:linearModel1} and \eqref{eq:linearModel2}, the identifiable parameters are 

\begin{equation}
	\boldsymbol{\beta} =
	\begin{bmatrix}
		\frac{a_f}{m} & \frac{a_f}{I_{xx}} & \frac{a_f}{I_{yy}} & \frac{b_f}{I_{zz}} & \Delta l_x & \Delta l_y 
	\end{bmatrix}^T, \hspace{10pt} \tau
	\label{eq:parameters}
\end{equation}

\noindent Then, is possible to rewrite the linear model in a more compact form:

\begin{equation}
	\begin{bmatrix}
		\ddot{\mathbf{x}}_B \\
		\dot{\boldsymbol{\omega}}_B
	\end{bmatrix}
	=
	\begin{bmatrix}
		L(\beta_1) \\
		A(\boldsymbol{\beta})
	\end{bmatrix}
	\mathbf{u}
	\label{eq:linearCompact}
\end{equation}

\noindent Under the assumption of the sampling rate to be much faster than the dynamics\footnote{In this case, thanks to the performance of the onboard electronics, the sampling rate is equal to $222$ Hertz.}, equation \eqref{eq:linearModel2} is implemented as discrete-time first order system, and the parameters are modeled as integrated white noise, which gives the following prediction equations

\begin{align}
	\boldsymbol{\omega}_{k} &= \boldsymbol{\omega}_{k-1}+\Delta t A(\boldsymbol{\beta}_{k-1})\mathbf{u}_{k-1} \\
    \mathbf{u}_k &= \frac{\tau_{k-1}}{\Delta t+\tau_{k-1}}\mathbf{u}_{k-1}+\frac{\Delta t}{\Delta t+\tau_{k-1}}\mathbf{u}_{in,k}^2 \\
    \boldsymbol{\beta}_k &= \boldsymbol{\beta}_{k-1} \\
    \tau_k &= \tau_{k-1}
\end{align}

\noindent where $\mathbf{u}_k$ and $\mathbf{u}_{in,k}$ are the inputs at time instant $k$ expressed in a vectorial way, $\Delta t$ is the sampling period and $\bullet^2$ is the element-wise square of a vector.


\section{Kalman filter}
\label{KalmanFilter}

A Kalman filter approach is chose for this project since it has good result in this kind of applications. Of course, better performances can be obtained with specific strategies for non linear systems \cite{parameterIdentification}, but these methods are in general much more complicated and require much more computational effort, especially if is necessary to estimate the parameters online.  

\noindent Now, is possible to use the standard Kalman filter equations \cite{KalmanFilter} to develop an online identification algorithm as follow.

\noindent The augmented state $\mathbf{x}_{est}$ is

\begin{equation}
	\mathbf{x}_{est}=
	\begin{bmatrix}
		\boldsymbol{\omega}_B & \mathbf{u}_{in} & \boldsymbol{\beta} & \tau
	\end{bmatrix}^T
	\in \rm I\!R^{15}
	\label{eq:augmentedState}
\end{equation} 

\noindent The initial values of $\boldsymbol{\omega}_B$ and $\mathbf{u}_{in}$ are know, so the state is initialized with these. Moreover, due to the parameters $\boldsymbol{\beta}$ and $\tau$ having a constraint to being positive, they are implemented as $\exp(\boldsymbol{\beta})$ and $\exp(\tau)$ to force positive results from the estimation, while the $\boldsymbol{\Delta l}$ are constrained to be within the propellers (the length of the arms is set to be equal to one, since the correct length is not necessary for the identification) which is implemented using a zero centered logistic function

\begin{equation}
	\frac{2}{1-\exp(-\boldsymbol{\Delta l})} -1
	\label{eq:logisticFunction}
\end{equation}

\noindent With the augmented state is possible to write a new state space system in discrete time with matrix $A_{est}$\footnote{For notation, $\mathbf{0}_{a\times b}$ is equal to a zero matrix with $a$ rows and $b$ columns,$\mathbf{1}_{a\times b}$ is equal to a ones matrix with $a$ rows and $b$ columns,$I_a$ is the identity matrix of dimension $a\times a$, and the vector $\mathbf{x}_{est,a:b}$ are the entries from $a$ to $b$ of the augmented state (is implicit that is consider at time $k$)}

\begin{align*}
	A_{\boldsymbol{\omega}, \mathbf{u}_{in}} &= 
	\left[
	\begin{array}{c}
		2\Delta te^{\beta_2}(\Delta l_y-1)\cdot\mathbf{u}^T \\
		\hline
		-2\Delta te^{\beta_3}(\Delta l_x+1)\cdot\mathbf{u}^T \\
		\hline
		-\sgn(\Omega_i)2\Delta te^{\beta_4}\cdot\mathbf{u}^T
	\end{array}
	\right] 
	& &\in \rm I\!R^{3\times 4} \\
	A_{\boldsymbol{\omega}, \boldsymbol{\beta}_1} &= \mathbf{0}_{3\times 1} & &\in \rm I\!R^{3\times 1}\\
	A_{\boldsymbol{\omega}, \boldsymbol{\beta}_{2}} &=
	\begin{bmatrix}
		\Delta te^{\beta_2}\bigg(\Delta l_y\sum\limits_{i=1}^{4}u_i^2-u_1^2-u_2^2+u_3^2+u_4^2\bigg) & 0 & 0
	\end{bmatrix}^T 
	& &\in \rm I\!R^{3\times 1}\\
	A_{\boldsymbol{\omega}, \boldsymbol{\beta}_{3}} &=
	\begin{bmatrix}
		0 & -\Delta te^{\beta_3}\bigg(\Delta l_y\sum\limits_{i=1}^{4}u_i^2+u_1^2-u_2^2-u_3^2+u_4^2\bigg) & 0
	\end{bmatrix}^T 
	& &\in \rm I\!R^{3\times 1}\\
	A_{\boldsymbol{\omega}, \boldsymbol{\beta}_{4}} &=
	\begin{bmatrix}
		0 & 0 & -\Delta te^{\beta_4}\sum\limits_{i=1}^{4}\sgn(\Omega_i)u_i^2
	\end{bmatrix}^T 
	& &\in \rm I\!R^{3\times 1}\\
	A_{\boldsymbol{\omega}, \boldsymbol{\beta}_{5}} &=
	\begin{bmatrix}
		0 & 0 & \Delta t
	\end{bmatrix}^T 
	& &\in \rm I\!R^{3\times 1}\\
	A_{\boldsymbol{\omega}, \boldsymbol{\beta}_{6:7}} &=
	\begin{bmatrix}
		0 & \Delta te^{\beta_2}\sum\limits_{i=1}^{4}u_i^2 \\ 
		-\Delta te^{\beta_3}\sum\limits_{i=1}^{4}u_i^2	& 0 \\
		0 & 0 
	\end{bmatrix} 
	& &\in \rm I\!R^{3\times 2}\\
	A_{\boldsymbol{\omega}, \boldsymbol{\beta}} &=
	\left[
	\begin{array}{c|c|c|c|c|c}
		A_{\boldsymbol{\omega}, \boldsymbol{\beta}_1} & A_{\boldsymbol{\omega}, \boldsymbol{\beta}_2} & A_{\boldsymbol{\omega}, \boldsymbol{\beta}_3} & A_{\boldsymbol{\omega}, \boldsymbol{\beta}_4} & A_{\boldsymbol{\omega}, \boldsymbol{\beta}_5} & A_{\boldsymbol{\omega}, \boldsymbol{\beta}_{6:7}}
	\end{array}
	\right]
	& &\in \rm I\!R^{3\times 7}\\
	A_{\mathbf{u}_{in}} &= \Big(1-\frac{\Delta t}{\Delta t+e^{\tau}}\Big)\cdot I_4 & &\in \rm I\!R^{4\times 4} \\
	A_{est,k} &=
	\left[ 
	\begin{array}{c}
		\begin{array}{c|c|c|c}
			I_{3} & A_{\boldsymbol{\omega}, \mathbf{u}_{in}} & A_{\boldsymbol{\omega}, \boldsymbol{\beta}} & \mathbf{0}_{3\times 1}
		\end{array} \\
		\hline
		\begin{array}{c|c|c}
			\mathbf{0}_{4\times 3} & A_{\mathbf{u}_{in}} & \mathbf{0}_{4\times 8}
		\end{array} \\
		\hline
		\begin{array}{c|c} 
			\mathbf{0}_{8\times7} & I_8
		\end{array}
	\end{array}
	\right]
	& &\in \rm I\!R^{15\times 15}
\end{align*}

\noindent and then use the Kalman filter equations in a recursive way \cite{KalmanFilter}

\begin{align*}
	P_k &= A_{est,k}\cdot P_{k-1}\cdot A_{est,k}^T + Q & \hspace{37pt} &\in \rm I\!R^{15\times 15} \\
	H_k &=
	\left[
		\begin{array}{c|c}
			I_3 & \mathbf{0}_{3\times 12} \\
			\hline
			\mathbf{0}_{1\times 3} & \begin{array}{c|c|c|c} 2e^{\beta_1}\cdot\mathbf{u}^T & \mathbf{0}_{1\times 2} & e^{\beta_1}\sum\limits_{i=1}^{4}u_i^2 & \mathbf{0}_{1\times 5} \end{array}
		\end{array}
	\right] & &\in \rm I\!R^{4\times 15} \\
	S_k &= H_k\cdot P_k\cdot H_k^T + R & &\in \rm I\!R^{4\times 4} \\
	K_k &= P_k\cdot H_k^T\cdot S_k^{-1} & &\in \rm I\!R^{15\times 4} \\
	P_k &= \big(I_{15} - K_k\cdot H_k\big)\cdot P_{k-1} & &\in \rm I\!R^{15\times 15} \\
	\mathbf{x}_{est,k} &= \mathbf{x}_{est,k-1} + K_k\cdot\Big(\begin{bmatrix}\boldsymbol{\omega} \\ \ddot{x}_z \end{bmatrix} - \begin{bmatrix}\mathbf{x}_{est,1:3} \\ e^{\beta_1}\cdot\mathbf{1}_{1\times 4}\cdot\mathbf{x}_{est,4:7}^2 \end{bmatrix}\Big) & &\in \rm I\!R^{15\times 1}
\end{align*}

\noindent where $P_k$ is the state update covariance matrix based on model, $H_k$ maps the measurement to the states, $S_k$ is the update measurement covariance, $K_k$ is the update Kalman gain, $Q\in\rm I\!R^{15\times 15}$ is the fixed covariance matrix and $R\in\rm I\!R^{4\times 4}$ the fixed measurement covariance matrix. Both $Q$ and $R$ are diagonal matrices.


\section{Results}
\label{resultsKalman}

\begin{figure}[h]
	\centering
 	\includestandalone[]{./chapter_3/plots/omega}
 	\caption{Measured and estimated angular rate, $\boldsymbol{\omega}_B$ and $\hat{\boldsymbol{\omega}}_B$.}
 	\label{fig:omegaKalman}		
\end{figure}

\noindent The estimator needs to be setup with specific process and measurement covariance $Q$ and $R$, and the starting state covariance $P_0$. The values for $Q$ and $P_0$ where found simply with a trial and error procedure, while $R$ was taken from the noise densities of the measured signals. In particular we measured a steady state position of the quadrotor, record the acceleration in the z axis $\ddot{x}_{B,z}$ and the angular rate $\boldsymbol{\omega}_B$, then by analyzing these data, a noise variance was extracted. Moreover, since in the augmented state $\mathbf{x}_{est}$ is present also an estimation of the angular rate $\hat{\boldsymbol{\omega}}$, to evaluate the quality of the estimation was also compare it with the measured angular rate. In this case the initial values of the state $\mathbf{x}_{est}$ were chose to be considerably different from a real value, just to show the performance of the estimator.

\begin{figure}[h]
	\centering
	\includestandalone[]{./chapter_3/plots/param_group_left}
	\includestandalone[]{./chapter_3/plots/param_group_right}
	\includestandalone[]{./chapter_3/plots/param_group_bottom}
	\caption{Estimated parameters $\boldsymbol{\beta}$ and $\tau$.}
	\label{fig:betaTauKalman}
\end{figure}

\noindent As is possible to see in figure \ref{fig:omegaKalman}, the estimation of the angular rate yields good results from the beginning for all the three axis. In figure \ref{fig:betaTauKalman} are instead plotted the estimations of all parameters $\boldsymbol{\beta}$ and $\tau$. The identification was performed with the cart fixed in one position. Is possible to see that for almost all parameters there is convergence after about $15$ seconds, while for the parameter $\frac{b_f}{I_{zz}}$ is necessary more time. This can be explain by observing the angular rate, in particular note that $\omega_z$ is almost zero for about the first 10 seconds, due to the particular trajectory of the vehicle. Of course, is not possible to perform system identification without excitation of the system and thats why the parameters depending on $\omega_z$ require more time to converge.

\noindent In conclusion the algorithm has been shown to work quite well, and for this application is not then necessary to use more sophisticated techniques.
